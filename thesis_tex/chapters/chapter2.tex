\chapter{The Carbon Disclosure Climate Survey}

\section{CDP Report Structure}
The CDP survey has $12$ primary sections, with each section containign a list of subsecitons with relevant questions. This is a list of the primary sections as well as a brief introduction to the types of questions present in the section:
\begin{itemize}
    \item \textbf{C0 - Introduction:} General description of the organization, along with information on where the company operates geographically, the currency used to report financial information, and the reporting boundary (whether it is financial or operational control), and the ISIN if the company is public
    \item \textbf{C1 - Governance:} Governance structures and processes related to climate change within the organization. It includes questions about board-level oversight of climate-related issues, the roles and responsibilities of management in addressing climate change, and how climate-related risks and opportunities are integrated into the company's overall governance framework. This section provides insight into the company's commitment to addressing climate change at the highest level of its organizational structure.
    \item \textbf{C2 - Risks and Opportunities:} Identification of processes that the organization uses to identify, assess, and respond to climate-related risks and opportunities. It includes questions regarding the definition of time horizons (short, medium, and long-term) for these risks and opportunities, and specific related details. This section aims to understand how the company perceives and manages potential impacts of climate change on its business, highlighting its approach to mitigating risks and capitalizing on new opportunities arising from the changing climate landscape.
    \item \textbf{C3 - Business Strategy:} This section examines the company's business strategy in relation to climate change. It explores whether the organization's strategy includes a transition plan that aligns with a 1.5°C world scenario, detailing the nature and publicly available aspects of this plan. The focus is on understanding how the company's strategy is designed to adapt to and mitigate climate-related issues, and how it plans to transition towards a lower-carbon, more sustainable business model. This section also looks at how feedback is collected from shareholders on the transition plan, emphasizing the integration of climate considerations into the core business strategy.
    \item \textbf{C4 - Targets and Performance:} This section presents an in-depth analysis of the company's specific emissions targets, including which emissions scopes are covered, targeted reduction percentages, and the current progress towards these goals. It also examines the emissions reduction initiatives that the company had active during the reporting year, detailing associated investments and the expected payback period. This part of the report effectively illustrates both the targets set by the company for reducing emissions and the concrete initiatives underway to achieve these objectives.
    \item \textbf{C5 - Emissions Methodology:} Insights into the company's emissions methodology, including any structural changes that may have occurred. It outlines the base year emissions against which progress is measured and explains the methodology employed to collect and report emission data. Additionally, it highlights the protocols and standards adhered to, ensuring the accuracy, consistency, and comparability of emissions data over time.
    \item \textbf{C6 - Emissions Data}: This section covers both Scope 1 and Scope 2 emissions. It includes relevant details, such as the categorization of Scope 2 emissions as either location-based or market-based. Additionally, the section provides insights into emission intensity per product, allowing for a detailed examination of emissions in relation to the company's products and operations. 
    \item \textbf{C7 - Emissions Breakdowns:} This is the section where company emissions (scope 1 and 2) are broken down into various sub-categories based on country/region, business division, and sector production activity. Most importantly, this section also provides a \underline{breakdown of changes in gross global emissions} (Scope 1 and 2 combined), and for each of them a specification of how your emissions compare to the previous year. \textbf{Emissions breakdowns will be further analyzed in the response variable discussion and linked to this section.}
    \item \textbf{C8 - Energy:} A comprehensive description of energy purchases and consumption, with a specification of renewable and non-renewable sources as well as consumption breakdowns by location. This section is useful in understanding whether the company is actively purchasing renewable energy and whether the firm's activity are energy intensive. 
    \item \textbf{C9 - Additional metrics:} Description of other sustainability related metrics such as investments in low carbon research and developement, transport technologies, and product/services
    \item \textbf{C10 - Verification:} This section provides a comprehensive description of the verification methodologies that the firm implements to verify and audit its emissions scopes. The report includes the proportion of verified emissions by scopes, verification standards and status.
    \item \textbf{C11 - Carbon Pricing:} Assesment of whether the company is subjected to a carbon tax and, if so, in which geographies and under which regimes along with a description of the percentage of emission scopes covered by the policy and the strategies that the company is implementing to comply with the regulations. Additionally, the section asks whether the company has an internal price of carbon and its related objectives.
    \item \textbf{C12 - Engagement:} Analysis of the company's effort to engage with its value chain to reduce carbon emissions. In particular, the company discloses which agents does the company collaborate with, whether the company requires suppliers to meet certain sustainability criteria, and whether the company engages with customers to drive awareness on climate related issues. Additionally, the company discloses whether it engages with policy makers in a way that could influence climate related policy, law, or regulation
    \item \textbf{Other Sections:} Sections beyond C$12$ are not relevant for the purpose of the thesis. They include details on biodiversity and signoff details among other metrics.
\end{itemize}

