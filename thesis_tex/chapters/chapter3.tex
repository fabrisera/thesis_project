\chapter{Data Sources}
\label{ch:data-sources}

\begin{keytakeaway}
  In this chapter, we present our three primary data sources: the \textbf{CDP Climate Change Response Questionnaire}, the \textbf{Worldscope Fundamental Core Items}, and the \textbf{Global Industry Classification Standard (GICS)}. The CDP Climate Survey data contains the response surveys from 2011 to 2022, the Worldscope database provides detailed standardized financials, and the GICS is a hierarchical classification system for industries.
\end{keytakeaway}

% \section{Introduction}

% \noindent The reports can be requested from the CDP website \cite{CDPMain2024}, they available both in PDF and a structured CSV format. The CSV files encompass the responses submitted to the CDP questionnaire, which form the foundation of the predictor and response variables for this thesis. My work is a continuation of the initial data processing that was executed by Climate and Sustainability Impact Lab \cite{HarvardD3Lab2024} and kindly shared with me in the form of various stata files and a repository containg both the raw and processed data, where the multiple sections had been extracted from the survey and aligned across different years. I then imported all the data and further processed it to create the training and tests sets for the following models. 


\section{Overview of Data Sources}

\subsection{Climate Change Response Questionnaire}
The main data source is a collection of all CDP climate surveys from all disclosing companies between 2011 and 2022. The data was partially cleaned and processed by the Climate and Sustainability Impact Lab \cite{HarvardD3Lab2024} before being shared with me. Organized by firm-years, each observation in the dataset corresponds to a specific firm reporting in a given year, and is structured in a panel format, having a unique id year pair to uniquely identify each entry along with its $130$ features \footnote{For a detailed breakdown of the CDP features, see the variable dictionary in Appendix \ref{sec:variable-dictionary}}. The original dataset contained $34,588$ firm-years across $11$ years. In the analysis we will control for financial and industry-specific predictors, and we will focus on public companies, which represent $71\%$ of the firm-years in the dataset. A detailed breakdown of the data cleaning process is provided in Section \ref{sec:data-cleaning-process-flowchart}.

% \subsubsection{Important Considerations on the CDP Data}
% \begin{itemize}
%     \item \textbf{Reporting year lag:} The data from a given year corresponds to the financial and operational data from the previous year. This was an important observation when merging the CDP data with other data sources, such as the Worldscope financial data.
%     \item \textbf{Data processing:} The original data processing entailed the extraction of multiple sections from the survey, which were then systematically aligned across different years, ensuring consistency across times and adjusting the format when the questions on the CDP surveyed changed or were slightly modified. It is important to note that the fact that some questions were not asked in some years, and that the questions were not always the same across years, is a significant challenge for the analysis which is specifically focused on forecasting emissions.

% \end{itemize}


\subsection{Worldscope Fundamental Core Items}
In addition to the CDP Climate Survey data, financial predictors were obtained using the Worldscope database \cite{Worldscope_2} accessed through Wharton Research Data Services (WRDS) \cite{WRDS}. Worldscope offers detailed standardized financials, allowing for comparisons of financial information across companies from various industries worldwide. This database boasts a long history, with over 35 years of data for key developed markets dating back to 1980 and more than 25 years for emerging markets. With its extensive coverage of over 100,000 companies in more than 120 countries, including full standardized coverage of over 30 developed and emerging markets and accounting for 99\% of global market capitalization, Worldscope is a comprehensive source for firm-level data. I queried the fundamental annuals features, which provide key global information such as \textit{revenue}, \textit{total assets}, \textit{number of employees}, and \textit{net income} \cite{Worldscope_2}. Data was retrieved based on the ISIN code, and resulted in $96\%$ of the firm-years having matching financial data. Of those, $17\%$ had missing values for at least one financial variable and the corresponding firm-years were dropped from the data-set. 

 \subsection{GICS}
 Accessed through WRDS using Capital IQ, the Global Industry Classification Standard (GICS) provides the framework for this study's industry analysis. GICS, a collaborative creation by MSCI and S\&P Dow Jones Indices, offers a hierarchical, four-tiered classification system, encompassing Sectors, Industry Groups, Industries, and Sub-Industries. This standard ensures a consistent approach to defining company activities worldwide, crucial for comparative financial analysis. The classification of a company within GICS is based on its principal business activity, with revenue being a primary determinant. I queried the GICS data only for the firm-years that had matching financial data, resulting in $19,200$ firm-years with complete financial and GICS data. GICS data was available for $99\%$ of the firm-years that had matching financial data. In our analysis, we utilize the 25 industry groups defined within GICS as categorical predictors.
 
 % insert a figure here
\begin{figure}[H]
\begin{center}
\includegraphics[width=5in]{figures/gics.png}
\caption{Global Industry Classification Standard (GICS) Structure \cite{GICS_MSCI}.}
\label{fig:label1}
\end{center}
\end{figure}

\section{Data Cleaning Process Flowchart}
\label{sec:data-cleaning-process-flowchart}

\noindent This is a visual representation of the data cleaning process described in the data sources with a specification of the number of firm-years dropped at each step: 

\bigskip

\begin{figure}[H]
\centering
\begin{forest}
    for tree={
        draw,
        rounded corners,
        align=left,
        edge={->},
        parent anchor=south,
        child anchor=north,
        l sep+=0.7cm, % Increase the level distance
        scale= 0.78,
    },
    [CDP Climate Response Survey Data\\\textbf{34,588 firm-years \& 11,289 firms}, fill=blue!20
        [Firm has an ISIN code?, edge, fill=yellow!20
            [\textbf{24,803 firm-years \& 4,330 firms}, edge label={node[midway, above left] {YES}}, fill=green!20
                [Is Worldscope + GICS data available?, edge, fill=yellow!20
                    [\textbf{19,200 firm-years \& 3,616 firms}, edge label={node[midway, above left] {YES}}, fill=green!20
                        [Is CDP control data available?, edge, fill=yellow!20
                            [\underline{\textbf{13,741 firm-years \& 3,574 firms}}, edge label={node[midway, above left] {YES}}, fill=green!20]
                            [\text{5,459} firm-years dropped, edge label={node[midway, above right] {NO}}, fill=red!20]
                        ]
                    ]
                    [\text{5,603} firm-years dropped, edge label={node[midway, above right] {NO}}, fill=red!20]
                ]
            ]
            [\text{9,785} firm-years dropped, edge label={node[midway, above right] {NO}}, fill=red!20]
        ]
    ]
\end{forest}
\captionof{table}{Data Cleaning Process Flowchart}
\label{fig:cdp-climate-response-survey-data}
\end{figure}

\noindent The final dataset contains \textbf{$\bf{13,741}$ firm-years across $\bf{3,574}$ firms, with complete CDP, Worldscope, and GICS data}. 

\section{Exploratory Data Analysis}

\subsection{The Response Variable: Next-Year Real Decarbonization Rate}
\label{sec:response-variable}

\begin{figure}[H]
    \begin{center}
    \includegraphics[width=5in]{figures/ghg_change_real_dist.png}
    \caption{Next Year Real Decarbonization Rate}
    \label{fig:next-year-real-decarbonization-rate}
    \end{center}
\end{figure}

As we can observe from Figure \ref{fig:next-year-real-decarbonization-rate}, the distribution our our response variable has three key features (i) a negative mean and a heavy left tail - that is on average firms are (gladly) reducing real emissions, with varying level of success (ii) some firms increase their real emissions, this is a minority but can be observed in the right tail and (iii) the mode is zero, that is there is a significant number of firm whose decarbonization rate is marked as $0$. This is a problematic aspect of the data as some firms do not make any change to their operations and renewable energy supply, thus having a true zero-change, while other firms do not provide a decarbonization rate even if a change in emissions occurred. In the following chapters, I will use the modeling results to discuss the impact of missing decarbonization rates and how to improve the CDP survey to mitigate this issue and distinguish between genuine zero changes and failure to disclose adequately.

\subsubsection{Real Decarbonization}

Our interest lies in determining which firms are actively reducing their carbon footprint through the adoption of advanced technology or the use of cleaner energy sources, rather than by relying on other indirect methods. Specifically, we aim for a reduction in global emissions levels; thus, strategies like divesting from a subsidiary or merely reducing production do not align with our objectives. We seek to identify and predict which firms are making significant changes to their operations to emit less and become more efficient. For this purpose, the real decarbonization rate —defined as the change in emissions attributable exclusively to process improvements and increased use of renewable energy— serves as the most appropriate response variable for our model. With this approach we exclude emissions reductions achieved through indirect methods such as carbon offsetting, focusing instead on genuine operational efficiencies and technological advancements.


% Out interest is in determining which firms are taking action to reduce their carbon footprint by adopting better technology or by utilizing cleaner energy in their processes, and not by acting on other indirect metrics, such as carbon offsetting. That is, we want the overall amount of emissions to go down across all firms, and therefore we are not interested if a firm reduces its emissions by divesting from a subsidiary, or by producing less. Rather, we want to identify and forecast which firms are taking actions that are going to enable them to emit less and operate more efficiently. In this regard, real decarbonization rate only takes into account change in emissions due to genuine operational efficiencies and technological advancements and renewable energy use, therefore is the response variable that is best aligned with our modeling objectives. 


\subsection{Number of Unique Reporting Firms by Year}
As can be observed from Figure \ref{fig:unique-companies-by-year}, there has been a noticeable increase in the number of unique firms reporting over time. It is important to distinguish that this figure represents firms within the selected training set, where a criterion for selection is at least three years of reporting. Thus, while the graph shows the growing trend of unique firms reporting, the actual total number of firms engaged in reporting is even higher. Therefore, not only the number of firms reporting to the CDP is increasing, but also the firms who do so over time. The growth in the number of reporting firms is a positive trend, as it expands the dataset available for building more accurate models and improving our capabilities in forecasting decarbonization rates.


\begin{figure}[H]
  \begin{center}
  \includegraphics[width=5in]{figures/unique_companies.png}
  \caption{Number of unique reporting firms by year}
  \label{fig:unique-companies-by-year}
  \end{center}
\end{figure}

\subsection{Real Decarbonization Rate Breakdown by Continent}
Figure \ref{fig:emission-breakdown-by-continent} and Table \ref{tab:emission-breakdown-by-continent} show the mean real decarbonization rate by continent across all CDP reporting years from 2011 to 2022. As expected, there is significant class imbalance between continents, with Europe having the most number of firms, followed by North America and Asia. There is a significant difference in the mean decarbonization rate across continents, with Europe having the best mean decarbonization rate with an average yearly decrease of $-4.94 \%$ and Africa having the worst mean decarbonization rate with an average yearly decrease of just $-2.81 \%$. Overall, the data suggests that operating in an environment with stricter mandates to report and reduce emissions, such as Europe, is associated with a higher mean decarbonization rate. This is consistent with the findings of Downar et al. \cite{Downar2020The} which show that firms with a carbon disclosure mandate reduced emissions by $8\%$ in a UK-based study.

\noindent 

\begin{figure}[H]
    \begin{center}
    \includegraphics[width=5in]{figures/mean_decarbonization_rate_by_continent.png}
    \caption{Mean Real Decarbonization Rate by continent}
    \label{fig:emission-breakdown-by-continent}
    \end{center}
\end{figure}
  

\begin{longtable}{lrlll}
\toprule
 & \# firms & Mean & Median & Std \\
Continent &  &  &  &  \\
\midrule
\endfirsthead
\toprule
 & \# firms & Mean & Median & Std \\
Continent &  &  &  &  \\
\midrule
\endhead
\midrule
\multicolumn{5}{r}{Continued on next page} \\
\midrule
\endfoot
\bottomrule
\endlastfoot
Africa & 79 & -2.8\% & -0.9\% & 5.9\% \\
Asia & 1175 & -3.1\% & -1.0\% & 6.4\% \\
Europe & 1217 & -4.9\% & -1.9\% & 8.6\% \\
North America & 933 & -3.5\% & -1.0\% & 7.3\% \\
Oceania & 101 & -3.1\% & -0.6\% & 6.9\% \\
South America & 90 & -4.0\% & 0.0\% & 10.1\% \\
\end{longtable}



\subsection{Real Decarbonization Rate Breakdown by Sector}

\noindent Figure \ref{fig:mean-decarbonization-rate-by-sector} and Table \ref{tab:Industry-Real-Decarbonizaton-Breakdown} show the mean real decarbonization rate by sector across all CDP reporting years from 2011 to 2022. The mean decarbonization rate varies significantly across sectors, with the best mean decarbonization rate in Software and Services, with an average yearly decrease of $-6.67 \%$, and the worst mean decarbonization rate in Materials sector, with an average yearly decrease of $-2.46\%$. Additionally, there are significant differences in the number of firms across sectors, with the Capital Goods sector having the most number of firms, $475$ and the Household and Personal Products sector having the least number of firms, $43$. Differences in sectors are important to consider, as they can be indicative of the difficulty of decarbonizing a given industry. For example, our data suggests that Transportation and Materials are the sectors with the worst mean decarbonization rates, consistent with the findings of Davis et al. which suggest that difficult-to-decarbonize energy services include aviation, long-distance transport, steel and cement production \cite{Davis2018Net-zero}.

\begin{figure}[H]
    \begin{center}
    \includegraphics[width=5in]{figures/mean_decarbonization_rate_by_sector.png}
    \caption{Mean Real Decarbonization Rate by Sector}
    \label{fig:mean-decarbonization-rate-by-sector}
    \end{center}
\end{figure}


\begin{longtable}{lrlll}
\toprule
 & \# firms & Mean & Median & Std \\
Sector &  &  &  &  \\
\midrule
\endfirsthead
\toprule
 & \# firms & Mean & Median & Std \\
Sector &  &  &  &  \\
\midrule
\endhead
\midrule
\multicolumn{5}{r}{Continued on next page} \\
\midrule
\endfoot
\bottomrule
\endlastfoot
Automobiles, Components & 124 & -3.2\% & -1.91\% & 5.54\% \\
Banks & 166 & -5.82\% & -2.9\% & 9.46\% \\
Capital Goods & 475 & -3.6\% & -1.1\% & 6.87\% \\
Commercial, Professional Services & 134 & -3.18\% & -0.04\% & 7.7\% \\
Consumer Durables, Apparel & 127 & -4.5\% & -1.4\% & 8.28\% \\
Consumer Services & 89 & -2.67\% & -0.96\% & 6.65\% \\
Consumer Staples Distribution, Retail & 65 & -3.72\% & -1.9\% & 6.72\% \\
Energy & 151 & -2.62\% & -0.15\% & 5.66\% \\
Equity Real Estate Investment Trusts & 96 & -5.53\% & -2.33\% & 8.9\% \\
Financial Services & 158 & -4.45\% & -0.6\% & 8.98\% \\
Food, Beverage, Tobacco & 187 & -3.75\% & -1.5\% & 6.63\% \\
Health Care Equipment, Services & 100 & -3.6\% & -0.9\% & 7.45\% \\
Household, Personal Products & 43 & -5.33\% & -2.4\% & 8.09\% \\
Insurance & 96 & -4.85\% & -2.0\% & 8.39\% \\
Materials & 420 & -2.46\% & -0.6\% & 5.78\% \\
Media, Entertainment & 70 & -4.93\% & -0.54\% & 8.25\% \\
Pharmaceuticals, Biotechnology, Life Sciences & 97 & -4.23\% & -1.8\% & 7.63\% \\
Real Estate Management, Development & 53 & -4.99\% & -1.1\% & 8.86\% \\
Retailing & 115 & -4.93\% & -1.3\% & 9.43\% \\
Semiconductors, Semiconductor Equipment & 79 & -3.6\% & -0.5\% & 8.15\% \\
Software, Services & 140 & -6.67\% & -2.85\% & 10.21\% \\
Technology Hardware, Equipment & 185 & -4.5\% & -1.8\% & 8.76\% \\
Telecommunication Services & 77 & -5.58\% & -2.34\% & 9.26\% \\
Transportation & 155 & -3.02\% & -1.0\% & 6.3\% \\
Utilities & 173 & -3.54\% & -0.1\% & 8.08\% \\
\caption{Real decarbonization rate breakdown by industry}
\label{tab:Industry-Real-Decarbonizaton-Breakdown}
\end{longtable}
 

\subsection{Real Decarbonization Rate Breakdown by Country}

Figure \ref{fig:mean-decarbonization-rate-by-country} shows the mean real decarbonization rate by country across all CDP reporting years from 2011 to 2022. We can observe are significant differences both in the number of firms per country and in the mean decarbonization rate. Table~\ref{tab:emission_breakdown_country} shows summary statistics for the worst $10$ performing countries with nonzero mean real decarbonization rates. Note how the median decarbonization rate is frequently zero, this is again an example of the missing data problem we introduced in the previous response variable, section \ref{sec:response-variable}.


% include graph of mean decarbonization rates by country
\begin{figure}[htbp]
    \begin{center}
    \includegraphics[width=5in]{figures/mean_decarbonization_rate_country.png}
    \caption{Mean Decarbonization Rate by Country}
    \label{fig:mean-decarbonization-rate-by-country}
    \end{center}
\end{figure}


\begin{longtable}{lrlll}
\toprule
 & \# firms & Mean & Median & Std \\
Country &  &  &  &  \\
\midrule
\endfirsthead
\toprule
 & \# firms & Mean & Median & Std \\
Country &  &  &  &  \\
\midrule
\endhead
\midrule
\multicolumn{5}{r}{Continued on next page} \\
\midrule
\endfoot
\bottomrule
\endlastfoot
Saudi Arabia & 1 & -0.6\% & -0.6\% & nan\% \\
Egypt & 2 & -1.46\% & 0.0\% & 2.85\% \\
Indonesia & 10 & -1.53\% & 0.0\% & 2.76\% \\
Malaysia & 13 & -1.57\% & 0.0\% & 6.19\% \\
Cayman Islands & 2 & -1.62\% & 0.0\% & 7.54\% \\
Peru & 1 & -1.67\% & 0.0\% & 4.53\% \\
China & 78 & -1.76\% & 0.0\% & 5.85\% \\
Hong Kong & 35 & -1.7\% & -0.27\% & 7.43\% \\
Philippines & 12 & -1.83\% & 0.0\% & 4.57\% \\
Thailand & 19 & -1.85\% & 0.0\% & 5.8\% \\
\caption{Emission Breakdown by Country}
\label{tab:emission_breakdown_country}
\end{longtable}


\subsection{Real Decarbonization Rate Breakdown by Year}

Figure \ref{fig:mean-decarbonization-rate-by-year} shows the mean and median real decarbonization rate by year across all CDP reporting years from 2011 to 2022. The data shows that the mean and median decarbonization rates have been (reassuringly) decreasing over time.

\begin{figure}[H]
    \begin{center}
    \includegraphics[width=5in]{figures/mean_decarbonization_rate_year.png}
    \caption{Mean, Median Real Decarbonization Rate by Year}
    \label{fig:mean-decarbonization-rate-by-year}
    \end{center}
\end{figure}

\begin{longtable}{lrlll}
\toprule
 & Count & Mean & Median & Std \\
Year &  &  &  &  \\
\midrule
\endfirsthead
\toprule
 & Count & Mean & Median & Std \\
Year &  &  &  &  \\
\midrule
\endhead
\midrule
\multicolumn{5}{r}{Continued on next page} \\
\midrule
\endfoot
\bottomrule
\endlastfoot
2011 & 1109 & -2.51\% & 0.0\% & 6.09\% \\
2012 & 1252 & -3.04\% & 0.0\% & 6.03\% \\
2013 & 1317 & -3.29\% & -1.0\% & 5.85\% \\
2014 & 1322 & -3.31\% & -1.3\% & 6.01\% \\
2015 & 1388 & -3.66\% & -1.7\% & 6.1\% \\
2016 & 1451 & -3.41\% & -1.44\% & 5.97\% \\
2017 & 1483 & -3.59\% & -1.5\% & 6.68\% \\
2018 & 1374 & -4.14\% & -1.55\% & 7.69\% \\
2019 & 1530 & -4.19\% & -1.2\% & 8.08\% \\
2020 & 1701 & -4.52\% & -1.5\% & 8.59\% \\
2021 & 2088 & -5.2\% & -1.2\% & 9.75\% \\
2022 & 2461 & -4.7\% & -0.82\% & 9.79\% \\
\caption{Real decarbonization rate by year}
\label{tab:real-decarbonization-rate-breakdown-by-year}
\end{longtable}



\section{Financial Predictors}
Figure \ref{fig:grid} shows the distribution of the financial predictors used in the analysis. This is a list of each predictor along with a brief description of how it was derived:
\begin{itemize}
    \item \textbf{Total Assets \ref{fig:total-assets}}: The total assets of the firm, which is a measure of the firm's size and the scale of its operations. Directly obtained from the Worldscope database and transformed using the natural logarithm $log(1 + \text{Total Assets})$.
    \item \textbf{Market Capitalization \ref{fig:market-capitalization}}: The market capitalization of the firm, which is a measure of the firm's size and the scale of its operations. Directly obtained from the Worldscope database and transformed using the natural logarithm $log(1 + \text{Market Cap})$.
    \item \textbf{Return on Equity \ref{fig:roe}}: The return on equity of the firm, which is a measure of the firm's profitability. Since the return on equity is a percentage which can be negative, the following transformation was used: $log(1 + \frac{\text{ROE}}{100})$.
    \item \textbf{Revenue \ref{fig:revenue}}: The total revenue of the firm, which is a measure of the firm's size and the scale of its operations. Directly obtained from the Worldscope database and transformed using the natural logarithm $log(1 + \text{Revenue})$.
    \item \textbf{Net Income \ref{fig:net-income}}: The net income of the firm, which is a measure of the firm's profitability. Directly obtained from the Worldscope database and transformed using the natural logarithm $log(1 + \text{Net Income})$.
    \item \textbf{Employees \ref{fig:employees}}: The total number of employees of the firm, which is a measure of the firm's size and the scale of its operations. Directly obtained from the Worldscope database and transformed using the natural logarithm $log(1 + \text{Employees})$.
    \item \textbf{Total Assets 1yr Growth \ref{fig:total-assets-1yr-growth}}: The one year growth of the total assets of the firm, which is a measure of the firm's growth. Directly obtained from the Worldscope database and since the growth can be negative, the following transformation was used: $log(1 + \frac{\text{Total Assets 1yr Growth}}{100})$.
    \item \textbf{Employees 1yr Growth \ref{fig:employees-1yr-growth}}: The one year growth of the total number of employees of the firm, which is a measure of the firm's growth. Directly obtained from the Worldscope database and since the growth can be negative, the following transformation was used: $log(1 + \frac{\text{Employees 1yr Growth}}{100})$.
    \item \textbf{Net Income over Assets \ref{fig:net-income}}: The net income of the firm over its total assets, which is a measure of the firm's profitability. The feature was calculated with the following formula $log(1 + \frac{\text{Net Income}}{\text{Total Assets}})$.
\end{itemize}

\begin{figure}[H]
\centering

% Row 1
\subcaptionbox{Total Assets\label{fig:total-assets}}{%
  \includegraphics[width=0.3\linewidth]{figures/financial_preds/tot_assets_dist.png}}%
\hfill % spacing between images
\subcaptionbox{Market Capitalization\label{fig:market-capitalization}}{%
  \includegraphics[width=0.3\linewidth]{figures/financial_preds/mkt_cap_dist.png}}%
\hfill % spacing between images
\subcaptionbox{Return on Equity\label{fig:roe}}{%
  \includegraphics[width=0.3\linewidth]{figures/financial_preds/roe_dist.png}}%

% Row 2
\subcaptionbox{Revenue\label{fig:revenue}}{%
  \includegraphics[width=0.3\linewidth]{figures/financial_preds/revenue_dist.png}}%
\hfill
\subcaptionbox{Net Income\label{fig:net-income}}{%
  \includegraphics[width=0.3\linewidth]{figures/financial_preds/net_income_dist.png}}%
\hfill
\subcaptionbox{Employees\label{fig:employees}}{%
  \includegraphics[width=0.3\linewidth]{figures/financial_preds/employees_dist.png}}%

% Row 3
\subcaptionbox{Tot. Assets 1yr Growth\label{fig:total-assets-1yr-growth}}{%
  \includegraphics[width=0.3\linewidth]{figures/financial_preds/assets_1yr_growth_dist.png}}%
\hfill % spacing between images
\subcaptionbox{Employees 1yr Growth\label{fig:employees-1yr-growth}}{%
  \includegraphics[width=0.3\linewidth]{figures/financial_preds/employees_1yr_growth_dist.png}}%
\hfill % spacing between images
\subcaptionbox{Net Income Over Assets\label{fig:net-income-over-assets}}{%
  \includegraphics[width=0.3\linewidth]{figures/financial_preds/net_income_over_assets_dist.png}}%



\caption{Financial Predictors}
\label{fig:grid}
\end{figure}