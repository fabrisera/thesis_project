% the abstract

% \noindent In our study, we analyze carbon disclosure data from the Carbon Disclosure Project (CDP) to forecast next-year decarbonization rates at the firm level. We use a dataset of 3,574 firms from the CDP Climate survey, spanning the years 2011 to 2022. We employ a variety of machine learning techniques, including Mixed Effect, Bayesian Ridge, and Gradent Boosting regression. We are one of the first to use corporate disclosure data to forecast decarbonization rates at the firm level, adopting a forward-looking framework to determine the significance of disclosure predictors.

% Among $130$ tested metrics, our findings reveal that the most important predictors of decarbonization are the firm's sector, country, average amount of emission targets across Scope 1 and 2, whether the firm reports Scope 2 market based emissions, whether the firm verifies Scope 3 emissions, and whether the firm uses a Marginal Abatement Cost Curve (MACC) to determine initiatives for reducing emissions. We also find that this set of predictors is consistent across different modeling techniques, indicating the robustness of our results and is particularly indicative for next-year decarbonization rates. Our results further suggest that decarbonization progresses at varying rates across different industries and geographical regions, with distinct firm characteristics significantly influencing these rates. 

% Finally, we find that CatBoost regression (gradient boosting) outperforms other models in predicting next-year decarbonization rates, and that Mixed Effect regression is the preferred model for striking a balance between interpretability and predictive power. Both models can be used as a benchmark for future research in corporate decarbonization forecasting at the firm level. These results are valuable for investors, policy makers, and companies, informing them about the crucial actions most informative of future decarbonization and the state of utilizing corporate carbon disclosure for forecasting at the individual firm level. Additonally, through our results, we provide suggestions for improving the design of future disclosure surveys and enhancing the predictive power of corporate disclosure scores. 

Companies can report carbon emissions via voluntary carbon disclosure surveys, a method gaining global traction. However, the capacity of these surveys to forecast future decarbonization efforts for individual firms is largely unexplored. This study examines carbon disclosure data from the Carbon Disclosure Project (CDP) Climate Survey to forecast decarbonization rates for 3,574 firms from 2011 to 2022. Using machine learning techniques including Mixed Effect, Bayesian Ridge, and Gradient Boosting regression, we are one of the first to leverage disclosure data for predictive analysis at the firm-level. We identify a set of crucial predictors for decarbonization, including the firm's sector, country, emission targets across Scope 1 and 2, reporting of Scope 2 market-based emissions, verification of Scope 3 emissions, and the application of a Marginal Abatement Cost Curve (MACC) for emission reduction initiatives. These predictors prove consistent across modeling techniques, highlighting our analysis's reliability. CatBoost regression stands out for its predictive accuracy, while Mixed Effect regression provides a balance between interpretability and predictive capability, both serving as benchmarks for future research. Overall, our study offers critical insights for investors, policymakers, and companies on firm-level actions that indicate future decarbonization, alongside recommendations to enhance the efficacy of future disclosure surveys and the predictive validity of corporate disclosure scores.